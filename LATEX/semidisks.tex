\documentclass[aip,jcp, amsmath, amssymb, reprint]{revtex4-1}

\usepackage{graphicx}% Include figure files
\usepackage{dcolumn}% Align table columns on decimal point
\usepackage{bm}% bold math
%\usepackage[mathlines]{lineno}% Enable numbering of text and display math
%\linenumbers\relax % Commence numbering lines

\usepackage[utf8]{inputenc}
\usepackage[T1]{fontenc}
%% Loads a Times-like font. You can also load
%% {newtxtext,newtxtmath}, but not {times}, 
%% {txfonts} nor {mathtpm} as these packages
%% are obsolete and have been known to cause problems.
\usepackage{mathptmx} 

\draft

\begin{document}

\title{Probing DNA Mono-nucleotide Self-Assembly: A Novel Coarse-Grained Model Exploration}
\date{\today}

\author{Mattia Trapella}
%\email[E-mail:]{ mattia.trapella@dottorandi.unipg.com}
\affiliation{Dipartimento di Fisica e Geologia, Universit\`a di Perugia, Perugia, Italia}

\author{Tommaso Bellini}
%\email[E-mail:]{ tommaso.bellini@unimi.it}
\affiliation{Dipartimento di Biotecnologie Mediche e Medicina Traslazionale, Universit\`a degli studi di Milano, Milano, Italia}
\author{Cristiano De Michele}
\email[E-mail:]{ cristiano.demichele@uniroma1.it}
\affiliation{Dipartimento di Fisica, “Sapienza” Universit\`a di Roma, Roma,Italia}

\begin{abstract}
  Origins of life are based on supra-molecular structuring
  of nucleic acid polymers in solution, which allows transfer and memory of genetic information in life. 
  Hence, study single mono-nucleotide self-assembly into DNA duplexes is crucial to understand 
  the emergence of life on earth.
 . %\cite{Jia}. 
  In this work, we numerically investigated the phase behaviour of a solution of DNA
  mono-nucleotides by developing a novel, extremely coarse-grained model. Specifically, 
  we designed a model of a single nucleotide based on a semi-disk-like polyhedron decorated with attractive 
  (patchy) sites for mimicking stacking and pairing interactions. Noteworthy, 
  this model successfully captured the experimental phase behavior which has been recently studied.
  We carried out Monte Carlo simulations of patchy polyhedra by adapting algorithms borrowed 
  from computer graphics.% \cite{libccd}.
\end{abstract}

\maketitle


\section{Introduction}
Understanding the liquid crystal phase of DNA is of paramount significance for unraveling the enigma of origins of life
on earth. Understanding the self-assembly of DNA mono-nucleotides not only could shed light on the genesis of 
supra-molecular structures crucial for genetic information transfer and memory in life, but also could help the 
development/validation of theories on the emergence of life itself~\cite{Jia}. 
For example it has been argued that  liquid crystal assemblies of nucleic acids may have facilitated the formation 
of long biopolymers necessary for life. In this context, the study of DNA liquid crystals may provide 
insights into the fundamental processes behind life emergence.  \\

Recent research has demonstrated a fascinating phenomenon in the behavior of nucleic acid mononucleotide triphosphates
(dNTPs) when subjected to specific conditions: at sufficiently high concentrations and low temperatures in an aqueous
solution, these molecules undergo a liquid crystalline transition to a columnar phase~\cite{Smith}. 
These dNTPs, with an approximate length of 4 nm, can stack one on top of each other, forming longer strands and thus 
inducing the formation of the columnar phase. These experimental findings calls for 
the development of computational model able to capture and understand the observed phase behavior.
In the literature~\cite{Nguyen,C0JM02355H,Bellini} 
coarse-grained models have been proposed which approximate these filaments as hard cylinders and quasi-cylinders, each with two
attractive patches. Anyway, these models exhibit a different phase behavior, in that a nematic liquid crystal is also formed
under suitable concentration and temperature, whereas this phase is not observed for dNTDs.
This inconsistency between experimental findings and simulations results prompts the
pursuit of a suitable numerical description of these experiments. 
The goal of the present work is precisely to develop a coarse-grained mode of dNTPs which fully accounts for the experimental 
results, which exhibit a transition from an isotropic to a columnar phase without any intermediate nematic phase and which is 
possible able to reproduce the temperature dependence of the phase boundaries.

\section{\label{sec:level1}Model and simulation details}
\begin{figure}[h!]
\includegraphics[width=1\linewidth]{rota.png}
\caption{\label{fig:rota} Semi-disk, that approximate the single nucleotides, with 4 attractive patches, two for stacking and two for pairing. The staking patches are rotated by $\pi/10$ so that the strand rotates on itself.}
\end{figure}
Our approach consists in modeling the dNTPs as semi-disks-like polyhedra 
featuring four distinct attractive patches. More specifically, two patches are placed along the cut
edge of the semi-disks to replicate the base-base pairing interaction, while two additional patch are located
at the center of the two bases of the semi-disk to mimic the stacking attraction between two dNTPs. 
Notably, to take into account the characteristic twist
of self-assembled DNA strands, the two stacking patches are 
placed in such a way that they give rise to their complete revolution
every 12 bases around the symmetry axis of an self-assembled DNA duplex. 
This is achieved by rotating the two patches belonging to a single semidisk by $\pi/10$ 
as shown in Figure~\ref{fig:rota}. 
% COMMENTO: se parto da una sistema di monomeri i dischi con patch 


Each semidisk is modeled as a polyhedra composed of a finite number of vertices, hence they are not exact semidisks (i.e. 
obtained by cutting in half a disk). Therefore, in the following we discuss a procedure to estimate the minimal number 
vertices to use to consider the polyhedra equivalent to exact semidisks. In this respect the goal was to have a polyhedra
which provides a physical behavior identical to that of exact semidisks. In a Monte Carlo simulation of polyhedra is needed
to have an algorithm for detecting their overlap. Since the number of vertices is not necessarily small, 
it was important to employ an efficient algorithm. After having done a comprehensive
evaluation, we found that the algorithm Xenocollide\cite{libccd} was best option, being both computationally very efficient 
and robust (i.e.~it is ensured that overlap are not missed).\\

%%%% TODO: sono arrivato qui
Furthermore, iterative adjustments to the model parameters are undertaken to ensure congruence with experimental data.
This includes fine-tuning the spatial placement and size of the attractive patches, as well as refining the dimensions
and rotational dynamics of the stacking patches. These refinements are imperative for achieving concordance between
simulation outputs and empirical measures such as the persistence length and the length of the DNA filament. 

\begin{figure}[h!]
\includegraphics[width=0.7\linewidth]{sosti4.png}
\caption{\label{fig:fila} The cylinder is replaced by 24 semi-disks. the cylinder patches are rotated to match once the semi-disks are replaced.}
\end{figure}

To expedite simulation processes, the liquid crystalline phases (namely, nematic and columnar) are initially discerned utilising four-patch cylinders. Subsequently, these cylinders are substituted with filamentous semi-disks, and the stability is ascertained through Monte Carlo NVT simulations. The equation of state is subsequently formulated, encompassing isotropic (I), nematic (N), and columnar (C) phases pertinent to four-patch cylinders, as illustrated in Figure 2. The adoption of four-patch cylinders facilitates the seamless integration of 24 semi-disk filaments, ensuring alignment of the final patches and the preservation of intermolecular bonds upon substitution, as shown in Figure \ref{fig:fila}.

\subsection{Polyhedra approximation}
The polyhedral approximation must be validated to ensure it maintains fidelity to the physics of the simulation.
Determination of an optimal number of vertices for polyhedra approximation, balancing computational efficiency and
accuracy, is crucial. Limiting the number of vertices is imperative to prevent excessive simulation times, using the
\textit{Xenocollide} algorithm, where execution time scales linearly with vertex count. Thus we investigate the equation
of state for hard cylinders, with and without polyhedra approximation, in an NPT ensemble at a constant temperature for
two cylinder elongations. Cylinders were approximated by polyhedra with 16, 24 and 32 vertices, followed by system
equilibration over four million Monte Carlo steps.
\begin{figure}[h!] \includegraphics[width=0.9\linewidth]{cylapprox.png}
  \caption{\label{fig:cylapprox} Equation of state of cylinders with aspect ratio of $L/D=3.68794$. The equation of
  state of the cylinders is compared with their approximation to polyhedra with 16, 24 and 32 vertices.}
\end{figure}

Figure~\ref{fig:cylapprox} illustrate the resulting equations of state, demonstrating preservation of system phase
transitions. A slight shift towards smaller volume fractions is observed, particularly notable with lower vertex counts.
However, satisfactory approximation is achieved with $V=32$ vertices. Thus, this vertex count is generally adopted
herein, unless explicitly specified otherwise.

\subsection{Stacking Free Energy}

In this paragraph, we discuss the calculation of the Stacking free energy~\cite{Michele}, which is essential for
transitioning from two-patch cylinders to four-patch cylinders. Indeed, when replacing the cylinders with cut disks in
nematic or columnar phases, the transition entails having one patch per base with two patches. Since the aim is to
maintain the bonding energy unchanged, it is imperative to enforce the same stacking free energy in both systems.
Stacking free energy is defined as:
\begin{equation}
	\beta\Delta F_b=\ln \bigg[2\frac{\Delta(T)}{v_d}\bigg]
	\label{deltaf}
\end{equation}

where:
\begin{equation}
\label{deltat}
	\Delta(T)=\frac{1}{4}\bigg\langle \int_{V_b} [e^{-\beta V(\textbf{r}_{12}, \Omega_1, \Omega_2)}-1]d\textbf{r}_{12}\bigg\rangle
\end{equation}

With $\textbf{r}$ denoting the vector connecting the centers of mass of particle 1 and particle 2, $\Omega_i$
representing the orientation of the i-th particle, and $\langle\ldots\rangle$ indicating the average taken over all
directions, $v_d$ stands for the volume of the considered object, while $V_b$ represents the bonding volume. The
calculation of $\Delta (T)$, in the case of attractive patches, is simplified because it suffices to calculate the
probability of obtaining a single bond and that of obtaining a double bond, then weigh them respectively by $\exp(\beta
U_0)$ and $\exp(2\beta U_0)$. Consequently, the equation for $\Delta (T)$ in Eq.~\eqref{deltat} becomes:
\begin{equation}
\label{deltat2}
	\Delta(T)=\frac{1}{4N}\big( N_1(e^{\beta u_0}-1)+N_2(e^{2\beta u_0}-1)\big)
\end{equation}

Where $N$ indicates the total number of insertions, $N_1$ and $N_2$ represent the occurrences of having one and two
bonds, respectively. Subsequently, a program is developed to insert a cylinder at the origin and then insert a second
cylinder with random position and orientation inside a box with dimensions sufficient to contain both. The number of
bonded patches is verified for each insertion, with a total of 5 billion insertions performed. Following the
simulations, the reduced temperature, defined as $T^*_{2 patch}=k_B T_{2 patch}/U_0$, is sought, such that 
${\Delta F_b}^{1 patch}(T^*_{1 patch})=\Delta F_b^{2 patch}(T^*_{2 patch})$, which is then utilised in simulations of cut disks to attain
the same stacking free energy.

\subsection{EOS cylinder with 4 patches}
Cylinders with 4 attractive patches are simulated to construct the equation of state, demonstrating the phase transition
between isotropic-nematic and nematic-columnar phases. The result is illustrated in Figure~\ref{fig:eos1}, where
isotropic phases are considered those with nematic order parameter $S<0.15$, and columnar phases are considered those
with hexagonal order parameter $\psi_6 > 0.3$. The nematic order parameter $S$ and the hexagonal order parameter $\psi_6
$ are calculated respectively by: 
\begin{equation}  
\label{pordnem}
S= \bigg\langle \frac{3}{2} \cos^2\theta - \frac{1}{2} \bigg\rangle; \; \; \; \; \; \; \; \; \; \langle \psi_6\rangle=\bigg\langle \frac{1}{N} \sum_{i=1}^N \frac{1}{n(i)} \sum_{j=1}^{n(i)}e^{6i\theta_{ij}} \bigg\rangle
\end{equation}


\begin{figure}[h!]
\includegraphics[width=0.86\linewidth]{eos.png}
\caption{\label{fig:eos1} Equation of state for cylinders with aspect-ratio $X_0=2$ and 2 attractive patches per base. The simulations employ $N = 180$ particles and evolve over 4 million Monte Carlo steps. The symbols in red represent the simulations used in the replacements with semi-disks.}
\end{figure}


The discrepancies observed in the equation of state can be ascribed to the constrained number of particles employed in the simulations; specifically, $N=180$ cylinders were utilised. This decision was made to mitigate computational expenses, which would escalate significantly with an increased value of $N$, given the necessity to replace each cylinder with 24 semi-disks.

\section{\label{exp}Experimental methods}




\section{\label{Results}Results}
To verify the stability of the nematic and columnar phases, by substituting the cylinders with semi-disks filaments, NVT
simulations are conducted. The substitution with filaments should be done without creating intersections between the
objects; hence the filament must be entirely contained within the original cylinder without any parts protruding outside
of it. The distance between the paired bases of the filament is chosen so that the final height matches that of the
cylinder being replaced. Additionally, the filament is constructed by introducing a rotation of $\pi/5$, between base
pairs, to ensure that the stacking patches are already connected. To represent the substitution process, the nematic
phase at $P^*=0.55$ is selected, and the substitution process is illustrated in Figure \ref{sosti1}.

\begin{figure}[h!]
\includegraphics[width=0.9\linewidth]{sosti1.png} 
\caption{\label{sosti1} (\textbf{a}) Nematic phase with cylinders at pressure $P^*=0.55$. (\textbf{b}) Nematic phase after replacement with filaments. } 
\end{figure} 

The depth of the pairing patch well remains to be defined. Experimentally, the stacking bond is approximately an order
of magnitude stronger than the pairing bond. However, this difference depends on various factors, such as the solution
in which the DNA filaments are found. Consequently, the same energy is assigned to both pairing and stacking bonds. This
decision is made because if the phase dissolution occurs even under these conditions, it will certainly happen with
weaker pairing bonds. To achieve the same energy for stacking and pairing bonds, considering that pairing bonds consist
of two patches, the depth of the pairing patch well is halved.\\
Three nematic phases and four columnar phases of cylinders with different volume fractions, and thus concentrations, are
selected as indicated in red in Figure~\ref{fig:eos1}, then the cylinders are replaced by filaments of 12 pairs of dNTPs
and their stability is studied using a Monte Carlo NVT simulation at various temperatures $T^*=[0.10:0.15]$. It is noted
that within the temperature range studied, the initially nematic phases are never stable, whereas the initially columnar
phases are stable at low temperatures. 
\begin{figure*}[t!] \includegraphics[width=0.7\linewidth]{finaleeng.png} \caption{\label{fig:wide}Phase diagram of
    dTTP/dATP DNA strands. The square and round symbols refer to the experimental results while the triangular symbols
  refer to the results obtained from the simulations. The orange area indicates where the columnar phase is present.}
\end{figure*}

 The result is illustrated in Figure~\ref{fig:par} through the nematic and columnar order parameters for two
 configurations, one starting with a nematic phase and one starting with a columnar phase. The pair distribution
 functions of the columnar case in the plane orthogonal to the alignment direction are also shown in the Figure~\ref{fig:par}; note the worsening of the hexagonal order as the temperature increases.


\begin{figure}[h!]
\includegraphics[width=0.99\linewidth]{par.png}
\caption{\label{fig:par} In purple, the nematic order parameter of the simulations that started from nematic phases at volume fraction $\phi=0.357$. In green the hexagonal order parameter of the simulations that started from columnar phases t volume fraction $\phi=0.483$. The graph shows the persistence of columnar phases, also highlighted by the graphs of the pair distribution function, and the dissolution of initially nematic phases.}
\end{figure}



\subsection{Comparative Analysis of Experimental and Simulation Results}


The simulation results are compared with the experimental findings regarding dTTP/dATP \cite{Smith}. Experimental outcomes are delineated using concentration, expressed in units of measure ($mg/ml$). Therefore, to align simulation results with experimental data, a conversion utilising Equation \ref{conc} is necessary. 
\begin{equation}
	c=\frac{N}{V}m_N
	\label{conc}
\end{equation}
In this equation, $c$ denotes concentration, $N$ represents the number of semi-disks within the simulation box's volume $V$, and $m_N$ denotes the molecular mass of a single nucleotide, measured in Daltons ($Da$). For dTTP and dATP filaments, respective molecular masses are $m_N=482.2\; Da$ and $m_N=491.2\; Da$. Since the experimental results pertain to a mixture of dTTP and dATP filaments, the average of the two molecular masses is taken: $m_N=486.7\; Da$. In Figure \ref{fig:wide}, experimental findings on dATP/dTTP filaments \cite{Smith} are juxtaposed with Monte Carlo simulation results of semi-disks conducted in this study. Simulations initialised from higher concentrations exhibit a phase transition, whereas such a transition is absent at lower concentrations. This observed behaviour aligns with the experimental trajectory. Moreover, elevating the simulation temperature leads to a transition from columnar phases to isotropic-columnar phases, characterised by partial ordering.





\section{\label{Dis}Conclusion}

Our study presents a novel coarse-grained model that successfully elucidates the liquid crystal phase behaviour of dTTP/dATP nucleotide solutions. By employing semi-disk-like polyhedra decorated with four attractive patches to mimic stacking and pairing interactions, our model accurately replicates experimental observations of columnar phases without transitioning into nematic phases, a discrepancy previously encountered in computational simulations. Through meticulous adjustments and validations, including the calculation of Stacking free energy and substitution processes, our model demonstrates robustness in capturing the intricate behaviours of DNA nucleotides. Furthermore, our simulations offer insights into the stability of nematic and columnar phases under varying concentrations and temperatures. Notably, the stability of columnar phases at low temperatures corroborates experimental findings and underscores the predictive capability of our model. The comparison between simulation results and experimental data on dTTP/dATP filaments further validates the efficacy of our approach, highlighting concordance in phase transitions observed at different concentrations. \\

Overall, our study not only advances our understanding of liquid crystal phase behaviour in DNA nucleotide solutions but also underscores the potential of computational modelling in elucidating complex molecular phenomena. Future research endeavours may leverage our coarse-grained model to explore additional nucleotide systems and unravel further intricacies underlying liquid crystal phase transitions in nucleic acid polymers.


\bibliography{bib_cit}
\end{document}
